\section{Vector Space}
\label{sec:vector-space}

\subsection{Definition}
\label{subsec:definition}

Let \( V \) be a non-empty set, and let \( \mathbb{K} \) be a field (like \(\mathbb{R}\)).

We say that \( V \) is a \textbf{vector space over} \( \mathbb{K} \) if:

\begin{itemize}
    \item There is a binary operation \textbf{addition}: \( +: V \times V \to V \) such that \( u + v \in V \) for all \( u, v \in V \).
    \item There is a \textbf{scalar multiplication}: \( \cdot: \mathbb{K} \times V \to V \) such that \( a \cdot u \in V \) for all \( a \in \mathbb{K} \) and \( u \in V \).
    \item Eight axioms hold for all \( u, v, w \in V \) and \( a, b \in \mathbb{K} \):
    \begin{enumerate}
        \item \textbf{Associativity}:
        \begin{equation}
            u + (v + w) = (u + v) + w
        \end{equation}
        \item \textbf{Commutativity}:
        \begin{equation}
            u + v = v + u
        \end{equation}
        \item \textbf{Existence of zero vector}:
        \begin{equation}
            u + 0 = u
        \end{equation}
        \item \textbf{Existence of additive inverse}:
        \begin{equation}
            u + (-u) = 0
        \end{equation}
        \item \textbf{Distributivity over vector addition}:
        \begin{equation}
            a(u + v) = au + av
        \end{equation}
        \item \textbf{Distributivity over scalar addition}:
        \begin{equation}
            (a + b)u = au + bu
        \end{equation}
        \item \textbf{Associativity of scalar multiplication}:
        \begin{equation}
            a(bu) = (ab)u
        \end{equation}
        \item \textbf{Multiplicative identity}:
        \begin{equation}
            1u = u
        \end{equation}
    \end{enumerate}

\end{itemize}

\subsection{What \textit{is} a vector space, really?}
\label{subsec:what-is-a-vector-space}

Once we formalize the idea of vector spaces as sets of \textit{objects that we can add and scale}, the
concept applies \textbf{not just to arrows in space}, but to many other things: functions, polynomials,
matrices, sequences, etc.

All those are \textbf{vector spaces} as long as they \textbf{satisfy the axioms above}.

Then, the cool part about vector spaces is that the \textbf{abstraction} lets us transfer methods
and results from one domain to another. In other words, if we can prove something about vector spaces in
general, we can apply it to all the specific cases we care about.
