\section{Vector Subspace}
\label{sec:vector-subspace}

\subsection{Definition}
\label{subsec:definition}

Let \( V \) be a vector space over a field \( \mathbb{K} \).
A subset \( W \subseteq V \) is a \textbf{vector subspace} of \( V \) if \( W \) itself is a vector
space under the same operations of vector addition and scalar multiplication defined in \( V \).

To be a subspace, a subset \( W \) must satisfy the following three conditions:
\begin{enumerate}
    \item \textbf{Non-emptiness}:
    \begin{equation}
        0 \in W
    \end{equation}
    \item \textbf{Closure under addition}:
    \begin{equation}
        u, v \in W \implies u + v \in W
    \end{equation}
    \item \textbf{Closure under scalar multiplication}:
    \begin{equation}
        u \in W, a \in \mathbb{K} \implies au \in W
    \end{equation}
\end{enumerate}